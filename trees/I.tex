\documentclass[12pt]{standalone}
\usepackage{times}
\usepackage{parsetree}
\usepackage{textcomp}
\pagestyle{empty}
%----------------------------------------------------------------------
% Node labels in CGEL trees are defined with \NL, which is defined so that
% \NL{Abcd}{Xyz} yields a label with the function Abcd on the top, in
% sanserif font, followed by a colon, and the category Xyz on the bottom.
\newcommand{\NL}[2]{\begin{tabular}[t]{c}\small\textsf{#1:}\\
#2\end{tabular}}
%----------------------------------------------------------------------
\begin{document}
\begin{parsetree}
(.Clause.
    (.\NL{Adjunct}{PP}.
    (.\NL{Head}{P}. `if'')
    (.\NL{Comp}{Clause}. 
    (~.\NL{Subj}{NP}. `your usage argument')
    (.\NL{Head}{VP}.
    (.\NL{Head}{V}. `is' )
    (.\NL{Comp}{VP-Coordination}.
    (~.\NL{Coordinate}{VP}. `well-reasoned')
    (~.\NL{Coordinate}{VP}.`motivated by real-world consequences' )
    (.\NL{Coordinate}{VP}.
    (.\NL{Marker}{Coordinator}.`and' )
    (~.\NL{Head}{VP}.`based on evidence' ))))))
	(.\NL{Subj}{NP}. 
	(.\NL{Head}{Nom}.
	(.\NL{Head}{N}. `it')))
	(.\NL{Head}{VP}.
	(.\NL{Head}{V}. `\textquoteright s' )
	(.\NL{Mod}{AdvP}. 
	(.\NL{Head}{Adv}. `not' ))
	(.\NL{PredComp}{NP}. 
	(.\NL{Head}{Nom}. 
	(.\NL{Head}{Adj}. `pedantry' )))
))
\end{parsetree}
\end{document}
