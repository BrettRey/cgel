
\documentclass[tikz,border=12pt]{standalone}
\usepackage[linguistics]{forest}
\usepackage{times}
\usepackage{xcolor}
\usepackage[T1]{fontenc}
\pagestyle{empty}
%----------------------------------------------------------------------
% Node labels in CGEL trees are defined with \Node, 
% which is defined so that \Node{Abcd}{Xyz} yields 
% a label with the function Abcd on the top, in small
% sanserif font, followed by a colon, and the category 
% Xyz on the bottom.
\newcommand{\Node}[2]{\small\textsf{#1:}\\{#2}}
% For commonly used functions this is defined with \(function)
\newcommand{\Head}[1]{\Node{Head}{#1}}
\newcommand{\Subj}[1]{\Node{Subj}{#1}}
\newcommand{\Comp}[1]{\Node{Comp}{#1}}
\newcommand{\Mod}[1]{\Node{Mod}{#1}}
\newcommand{\Det}[1]{\Node{Det}{#1}}
\newcommand{\PredComp}[1]{\Node{PredComp}{#1}}
\newcommand{\Crd}[1]{\Node{Coordinate}{#1}}
\newcommand{\Mk}[1]{\Node{Marker}{#1}}
\newcommand{\Obj}[1]{\Node{Obj}{#1}}
\newcommand{\Sup}[1]{\Node{Supplement}{#1}}
%----------------------------------------------------------------------
\begin{document}
\begin{forest}
where n children=0{% for each terminal node
    font=\itshape, 			% italics
    tier=word          			% align at the "word" tier (bottom)
  }{%								% no false conditions, so empty
  },
[Clause
	[\Head{Clause}
		[\Subj{NP}[\Head{Nom}[\Head{N\textsubscript{\textsc{pro}}}[I]]]]
		[\Head{VP}
			[\Head{V}[don't]]
			[\Comp{Clause}
				[\Head{VP}
					[\Head{V}[think]]
					[\Comp{Clause}
						[\Subj{NP}[\Head{Nom}[\Head{N\textsubscript{\textsc{pro}}}[it]]]]
						[\Head{VP}
							[\Head{V}[will]]
							[\Comp{Clause}
								[\Head{VP}
									[\Head{V}[be]]
									[\PredComp{AdjP}[\Head{Adj}[hard]]]
								]
							]
						]
						[\Node{Extraposed Subject}{Clause}
							[\Head{VP}
								[\Mk{Sdr}[to]]
								[\Head{VP}
									[\Head{V}[keep]]
									[\Obj{NP}[\Det{NP}[\Head{Nom}[\Head{N\textsubscript{\textsc{pro}}}[his]]]][\Head{Nom}[\Head{N}[weight]]]]
									[\Comp{PP}[\Head{P}[up]]]
								]
							]
						]
					]
				]
			]
		]
	]
	[\Mod{PP}[\Head{P}[so long as]] %CGEL p. 1134 'as long as'  has lost its comparative meaning and been reanalysed as a compound preposition meaning “provided”
		[\Comp{Clause}
			[\Subj{NP}[\Head{Nom}[\Head{N\textsubscript{\textsc{pro}}}[he]]]]
			[\Head{VP}
				[\Head{V}[can]]
				[\Comp{Clause}
					[\Head{VP}
						[\Head{V}[eat]]
						[\Obj{NP}[\Det{DP}[\Head{D}[the]]][\Head{Nom}[\Head{N}[grain]]]]
					]
				]
			]
		]
	]
]
\end{forest}
\end{document}

