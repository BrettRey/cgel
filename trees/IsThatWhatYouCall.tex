\documentclass[12pt]{standalone}
\usepackage{times}
\usepackage{parsetree}
\pagestyle{empty}
%----------------------------------------------------------------------
% Node labels in CGEL trees are defined with \NL, which is defined so that
% \NL{Abcd}{Xyz} yields a label with the function Abcd on the top, in
% sanserif font, followed by a colon, and the category Xyz on the bottom.
\newcommand{\NL}[2]{\begin{tabular}[t]{c}\small\textsf{#1:}\\
#2\end{tabular}}
%----------------------------------------------------------------------
\begin{document}
\begin{parsetree}
(.Clause.
	(.\NL{Prenucleus}{V\textsubscript{x}}. `is' )
	(.\NL{Head}{Clause}.
	(.\NL{Subj}{NP}.
	(.\NL{Head}{Nom}.
	(.\NL{Deteterminer-Head}{DP}.
	(.\NL{Head}{D}. `that' ))))
	(.\NL{Head}{VP}.
	(.\NL{Head}{GAP\textsubscript{x}}. `--')
	(.\NL{PredComp}{NP}.
	(.\NL{Head}{Nom}.
	(.\NL{Modifier}{Clause\textsubscript{rel}}.
	(.\NL{Head-Prenucleus}{NP\textsubscript{y}}. `what' )
	(.\NL{Head}{Clause}.
	(~.\NL{Subj}{NP}. `you')
	(.\NL{Head}{VP}.
	(.\NL{Head}{V}. `call')
	(.\NL{Obj\textsubscript{dir}}{GAP\textsubscript{y}}. `--')
	(~.\NL{Obj\textsubscript{ind}}{NP}. `WH-movement')
	))))))))
\end{parsetree}
\end{document}
